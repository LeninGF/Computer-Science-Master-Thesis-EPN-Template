\section{Introduction}

\lipsum[4] \\

\section{Results}
Time to present tabular information

\begin{table}[h!]
  \begin{center}
    \caption{Inbreast Transfer Learning Results on Dataset $D_{3}$}
    \label{tab:inbreast-tl-d3}
    \pgfplotstabletypeset[
      multicolumn names, % allows to have multicolumn names
      col sep=comma, % the seperator in our .csv file
      display columns/0/.style={
		column name=Algorithm, % name of first column
		column type={c},string type},  % use siunitx for formatting
      display columns/1/.style={
		column name=Metric 1,
		column type={c},string type},
	display columns/2/.style={
		column name=Metric 2,
		column type={c},string type},
	display columns/3/.style={
		column name=Metric 3,
		column type={c},string type},
      every head row/.style={
		before row={\toprule}, % have a rule at top
		after row={
% 			\si{\ampere} & \si{\volt}\\ % the units seperated by &
			\midrule} % rule under units
			},
		every last row/.style={after row=\bottomrule}, % rule at bottom
    ]{csv/results_demo.csv} % filename/path to file
    % transfer-learning-results-top5_raw_roi_not_augm.csv
  \end{center}
\end{table}


\section{Experimental Results Discussion}
Please do not leave without discussing your findings.\lipsum[5]